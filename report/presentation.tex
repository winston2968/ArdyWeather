\documentclass{beamer}


% ========================================================================================
%                                   Setting up document
% ========================================================================================

% Use personnl theme
\usepackage{beamerthemeChampollion}

% Tikz package for drawing 
\usepackage{tikz}
\usetikzlibrary{positioning, shapes.geometric, arrows.meta}


% Title and logo
\title[]{Soutenance Mini-Projet Arduino}
\subtitle{ArdyWeather}
\author{Axel PIGEON}
\institute{INU Champollion}
\date{\today}
\titlegraphic{\includegraphics[height=1cm]{./images/logo_jfc.png}}



% ========================================================================================
%                                   CONTENT
% ========================================================================================

\begin{document}

% Titlepage 
{
    % \usebackgroundtemplate{\includegraphics[width=\paperwidth]{./images/background_titlepage.png}}
    \begin{frame}
        \titlepage
    \end{frame}
}

% ========================================================================================
% Introduction

\section{Introduction}

\begin{frame}{Introduction}
    \begin{minipage}{0.40\textwidth}
        \centering
        \includegraphics[width=1.0\textwidth]{./images/logo_arduino.png}
    \end{minipage}
    \hfill 
    \begin{minipage}{0.55\textwidth}
        \begin{itemize}
            \item Mini Projet Arduino 
            \item Architecture des Ordinateurs, L3 Informatique
            \item Objectif : communication entre Arduino
            \item \href{https://github.com/winston2968/ArdyWeather.git}{Lien GitHub}
        \end{itemize}
    \end{minipage}
\end{frame}

% Summary 
\begin{frame}{Plan}
    \tableofcontents
\end{frame}

% ========================================================================================
% Entreprise et missions

\section{Principe et fonctionnement}

\begin{frame}{Idée du projet}
    \centering
    \begin{tikzpicture}[node distance=2cm and 3cm]
        % Station node 
        \node[draw, ellipse] (s) at (0,4) {Station}; 
        % Sensors node
        \node[draw, ellipse] (c1) at (-4,1) {Capteur 1};
        \node[draw, ellipse] (c2) at (0,-2) {Capteur 2};
        \node[draw, ellipse] (c3) at (4,1) {Capteur 3};
        
        % Node links 
        \draw[->, dashed, bend left] (c1) to (s);  
        \draw[->, dashed, bend left] (s) to (c1); 
        \draw[->, dashed, bend left] (c2) to (s);  
        \draw[->, dashed, bend left] (s) to (c2);
        \draw[->, dashed, bend left] (c3) to (s);  
        \draw[->, dashed, bend left] (s) to (c3);
    \end{tikzpicture}
\end{frame}

\begin{frame}{Structure des paquets}
    $\Longrightarrow$ Besoin d'une structure de paquets avec : 
    \begin{itemize}
        \item {Une séquence propre au protocole}
        \item {Un identifiant unique pour l'émeteur}
        \item {Un identifiant unique pour le destinataire}
        \item {Le type du paquet}
        \item {Le type des données envoyées}
        \item {Le nombre de paquets déjà envoyés}
        \item {Le nombre de paquets reçus}  
        \item {Deux octets de checksum }
        \item {Les données}
    \end{itemize}
\end{frame}


\begin{frame}{Structure des paquets (2)}
    D'où le paquet suivant : 
        \begin{center}
            \texttt{\{'A', 'W', Emetteur, Destinataire, Type de Paquet, Type de Données, Nb Envoyés, Nb ACK, CHKSM1, CHKSM2, Datas\}}
        \end{center}
    Par exemple \texttt{'A','W','1','0','D','H','1','0',... }
\end{frame}

\begin{frame}{Encodage des données}
    \centering 
    Idée : \texttt{float table[4][16]}  $\longrightarrow$ 3 char ASCII dans $[33, 126]$
    
    \raggedright

    \vspace{0.7cm}


    Soit  $ V = \texttt{int}(100 \times \texttt{table[i][j]}) $ on pose alors : 
    $ V = a \cdot 91 + b \cdot 10 + c$ avec :
        \[ 
            \begin{cases}
                a = \left\lfloor \frac{V}{91} \right\rfloor \\
                b = \left\lfloor \frac{V \bmod 91}{10} \right\rfloor \\
                c = V \bmod 10 \\
            \end{cases}
            \Longrightarrow
            \begin{cases}
                c_1 = a + 33 \quad (\text{caractère ASCII}) \\
                c_2 = b + 33 \\
                c_3 = c + 33
            \end{cases}
        \]
\end{frame}

\begin{frame}{Calcul plage d'encodage}
    On encode un entier \texttt{value} (issu de \texttt{float}) sur 3 ASCII tel que : 
    \[ 
        \begin{cases}
            a = \texttt{value} \times 91 \Longrightarrow \max(a) = 126 - 33 = 93 \\ 
            b = (\texttt{value} \mod 91) / 10 \Longrightarrow \max(b) = 126 - 33 = 93 \\ 
            c = \texttt{value} \mod 10  \Longrightarrow \max(c) = 9
        \end{cases}
    \] 
    Tel que $ \texttt{value} = a \times 91 + b \times 10 + c $. 
    Avec : 
        \[ a \in [0, 93], \quad b \in [0, 9], \quad c \in [0, 9] \] 
    Donc : 
        \[ \texttt{value}_{\text{max}} = 8562 \quad \text{ i.e } \; 85.62 \] 
    
\end{frame}

% ========================================================================================
% Démonstration

\section{Démonstration}

\begin{frame}{Démonstration}
    \centering 
    \includegraphics[width=0.3\textwidth]{images/loading_icon.jpg}
\end{frame}

% ========================================================================================
% Conclusion

\section{Conclusion}

\begin{frame}{Conclusion}
    \begin{itemize}
        \item[-] Difficulté lié au peu de ressources disponibles. 
        \item[-] Langage C difficile. 
        \item[-] Capteurs Temp/Hum pas intégrés. 
        \item[+] Projet très intéressant.  
        \item[+] Développement d'un protocole de communication. 
        \item[+] Thème Beamer pour Champollion réussis ! 
    \end{itemize}
\end{frame}

\begin{frame}{Questions ?}
    \centering
    \huge Merci pour votre attention !
\end{frame}

\end{document}
