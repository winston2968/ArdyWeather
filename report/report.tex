\documentclass[a4paper]{article}

% HEADER


\usepackage[top=3.5cm, bottom=3cm, left=3.5cm, right=3.5cm]{geometry}


\setlength{\parindent}{0cm}


\usepackage[english]{babel}
\usepackage[T1]{fontenc}
\usepackage{ragged2e}
\usepackage{amsfonts}
\usepackage{systeme}
\usepackage{amsmath}
\usepackage{amssymb}
\usepackage{comment}
\usepackage{multicol}
\usepackage{lipsum} 
\usepackage{graphicx}
\usepackage{stmaryrd}
\usepackage{systeme}
\usepackage{wrapfig}
\usepackage{colortbl}
\usepackage{cellspace}
\usepackage{stmaryrd}
\usepackage{ntheorem}
\usepackage{lmodern}
\usepackage{mathtools}
\usepackage{ragged2e}
\usepackage{tabularx}
\usepackage{titlepic}
\usepackage{fancyhdr}
\usepackage{caption}
\usepackage{xcolor} % pour les couleurs
\usepackage[linkbordercolor=white]{hyperref} % après avoir chargé xcolor
\usepackage{systeme}
\usepackage[T1]{fontenc}
\usepackage{lmodern}
\usepackage{listings}
\usepackage{tikz}
\usepackage{mdframed}
\usepackage{xparse} % Nécessaire pour définir des environnements avec arguments optionnels
\usepackage{booktabs}  % Pour des tableaux plus jolis
\usepackage{tocloft}



% PAGE SETTINGS

% \justifying


\newcolumntype{C}{>{$\displaystyle}Sc<$}
\cellspacetoplimit=5pt
\cellspacebottomlimit=5pt

\setcounter{tocdepth}{1} % N'affiche que les sections (niveau 1)
\setlength{\cftsecnumwidth}{2em} % Largeur des numéros de section
\setlength{\cftsecindent}{0em}   % Pas d'indentation pour les sections

% MATHS SHORTHANDS

\newcommand{\C}{\mathbb{C}}
\newcommand{\R}{\mathbb{R}}
\newcommand{\Q}{\mathbb{Q}}
\newcommand{\Z}{\mathbb{Z}}
\newcommand{\N}{\mathbb{N}}
\newcommand{\U}{\mathbb{U}}
\newcommand{\K}{\mathbb{K}}
\newcommand{\M}{\mathcal{M}}
\newcommand{\B}{\mathcal{B}}

\renewcommand{\epsilon}{\varepsilon}
\renewcommand{\phi}{\varphi}
\renewcommand{\rho}{\varrho}

% MOD NOTATION

\theorembodyfont{\upshape}

% Définir un environnement pour encadrer les définitions avec un titre
\NewDocumentEnvironment{definition}{O{}}
{
  \begin{mdframed}[linewidth=0pt,linecolor=gray,backgroundcolor=gray!10,roundcorner=5pt]
  \textbf{Définition}%
  \IfNoValueTF{#1}{}{~(\textbf{#1})} % Affiche le titre entre parenthèses et en gras s'il est fourni
  . % Point à la fin
}
{
  \end{mdframed}
}

% Définir un environnement pour encadrer les théorèmes avec un titre
\NewDocumentEnvironment{theorem}{O{}}
{
  \begin{mdframed}[linewidth=1pt,linecolor=darkgray,backgroundcolor=darkgray!10,roundcorner=5pt]
  \textbf{Théorème}%
  \IfNoValueTF{#1}{}{~(\textbf{#1})} % Affiche le titre entre parenthèses et en gras s'il est fourni
  . % Point à la fin
}
{
  \end{mdframed}
}


\theoremstyle{plain}
\newtheorem*{remark}{Remarque}
\newtheorem*{proposition}{Proposition}
\newtheorem*{lemma}{Lemme}
\newtheorem*{prop}{Propriété}
\newtheorem*{corollary}{Corollaire}
\newtheorem*{proof}{Démonstration}
\newtheorem*{example}{Exemple}


% OTHERS 


\newlength\tindent
\setlength{\tindent}{\parindent}
\setlength{\parindent}{0pt}
\renewcommand{\indent}{\hspace*{\tindent}}


\usepackage{fancyvrb}
\usepackage{tikz-cd} 
\usepackage{amsmath}
\usepackage{mathrsfs}  
\usepackage{amssymb}
\usepackage{tkz-graph}
\usepackage{caption}
\usepackage{multicol}
\usepackage{listings} % Importation du package listings
\usepackage{xcolor} % Pour ajouter de la couleur

\setlength{\columnsep}{1cm} % Espace entre les colonnes


% Définition du style de l'en-tête
\pagestyle{fancy}
\fancyhf{} % Nettoyer les en-têtes et pieds de page

% Gauche : Nom de la section
\fancyhead[L]{\nouppercase{\leftmark}}

% Droite : Logo
\fancyhead[R]{\includegraphics[width=2cm]{./images/logo_jfc.png}}

% Ligne sous l'en-tête
\renewcommand{\headrulewidth}{0.4pt}

\setlength{\headheight}{1.5cm} % Ajuste la hauteur de l'en-tête





\begin{document}

% ==================================================================================================================================
% TITLEPAGE 

\begin{titlepage}
    \begin{center}
        \includegraphics[width=0.4\textwidth]{./images/logo_arduino.png} \\[1cm]

        {\Huge \textbf{Rapport Mini Projet Arduino}} \\[1cm]

        {\Large \textbf{ArdyWeather}} \\[2cm]

        {\textbf{UE : Architecture des Ordinateurs}} \\[2cm]

        \textbf{Auteur :} Axel PIGEON \\[0.5cm]
        \textbf{Université :} INU Champollion \\[0.5cm]
        \textbf{Date :} \today \\[2cm]

        \vfill

        {\large Rapport dans le cadre d'un projet de développement en Arduino ayant pour but 
        l'initiation au développement embarqué et la communication entre différentes architectures 
        (Arduino, RaspberryPI, etc...)}
    \end{center}
\end{titlepage}


\tableofcontents

% ==================================================================================================================================
% Introduction 

\justify

\subsection{Introduction}


Ici est présenté le projet d'Architecture des Ordinateurs 2. Le projet consiste à développer un outil ou un jeu 
utilisant des connexion arduino/arduino ou arduino/raspberry. 

Pour celui-ci, nous allons essayer de faire une station météo composée de deux modules : 
\begin{itemize}
    \item \textbf{Le capteur : } composé d'une carte arduino munie de capteurs (ex : température, humidité...) et d'un émetteur/récepteur 
    radio pour envoyer les relevés à la station fixe. 
    Pour le capteur, on utilise un {capteur de température et d'humidité Grove} permettant 
    une connexion facile à la carte. 
    \item \textbf{La station fixe : } Composée d'un arduino recevant les données émises par le capteur et les envoie en filaire 
    à un carte type raspberry pour un affichage sur écran/interface web. 
\end{itemize}

Toutes les informations techniques relatives au projet (code, bibliothèques) sont disponibles sur le github : 
\begin{center}
    \href{https://github.com/winston2968/ArdyWeather.git}{https://github.com/winston2968/ArdyWeather.git}
\end{center}

L'objectif est de programmer la station pour qu'elle puisse interragir avec plusieurs 
stations. 


\newpage 

% ==================================================================================================================================
% Présentation et principe général

\section{Présentation et Principe général}






% ==================================================================================================================================
% Développement

\section{Développement}







% ==================================================================================================================================
% Conclusion

\section{Conclusion}



\end{document}
