\documentclass[a4paper]{article}

% HEADER


\usepackage[top=3.5cm, bottom=3cm, left=3.5cm, right=3.5cm]{geometry}


\setlength{\parindent}{0cm}


\usepackage[english]{babel}
\usepackage[T1]{fontenc}
\usepackage{ragged2e}
\usepackage{amsfonts}
\usepackage{systeme}
\usepackage{amsmath}
\usepackage{amssymb}
\usepackage{comment}
\usepackage{multicol}
\usepackage{lipsum} 
\usepackage{graphicx}
\usepackage{stmaryrd}
\usepackage{systeme}
\usepackage{wrapfig}
\usepackage{colortbl}
\usepackage{cellspace}
\usepackage{stmaryrd}
\usepackage{ntheorem}
\usepackage{lmodern}
\usepackage{mathtools}
\usepackage{ragged2e}
\usepackage{tabularx}
\usepackage{titlepic}
\usepackage{fancyhdr}
\usepackage{caption}
\usepackage{xcolor} % pour les couleurs
\usepackage[linkbordercolor=white]{hyperref} % après avoir chargé xcolor
\usepackage{systeme}
\usepackage[T1]{fontenc}
\usepackage{lmodern}
\usepackage{listings}
\usepackage{tikz}
\usepackage{mdframed}
\usepackage{xparse} % Nécessaire pour définir des environnements avec arguments optionnels
\usepackage{booktabs}  % Pour des tableaux plus jolis
\usepackage{tocloft}



% PAGE SETTINGS

% \justifying


\newcolumntype{C}{>{$\displaystyle}Sc<$}
\cellspacetoplimit=5pt
\cellspacebottomlimit=5pt

\setcounter{tocdepth}{1} % N'affiche que les sections (niveau 1)
\setlength{\cftsecnumwidth}{2em} % Largeur des numéros de section
\setlength{\cftsecindent}{0em}   % Pas d'indentation pour les sections

% MATHS SHORTHANDS

\newcommand{\C}{\mathbb{C}}
\newcommand{\R}{\mathbb{R}}
\newcommand{\Q}{\mathbb{Q}}
\newcommand{\Z}{\mathbb{Z}}
\newcommand{\N}{\mathbb{N}}
\newcommand{\U}{\mathbb{U}}
\newcommand{\K}{\mathbb{K}}
\newcommand{\M}{\mathcal{M}}
\newcommand{\B}{\mathcal{B}}

\renewcommand{\epsilon}{\varepsilon}
\renewcommand{\phi}{\varphi}
\renewcommand{\rho}{\varrho}

% MOD NOTATION

\theorembodyfont{\upshape}

% Définir un environnement pour encadrer les définitions avec un titre
\NewDocumentEnvironment{definition}{O{}}
{
  \begin{mdframed}[linewidth=0pt,linecolor=gray,backgroundcolor=gray!10,roundcorner=5pt]
  \textbf{Définition}%
  \IfNoValueTF{#1}{}{~(\textbf{#1})} % Affiche le titre entre parenthèses et en gras s'il est fourni
  . % Point à la fin
}
{
  \end{mdframed}
}

% Définir un environnement pour encadrer les théorèmes avec un titre
\NewDocumentEnvironment{theorem}{O{}}
{
  \begin{mdframed}[linewidth=1pt,linecolor=darkgray,backgroundcolor=darkgray!10,roundcorner=5pt]
  \textbf{Théorème}%
  \IfNoValueTF{#1}{}{~(\textbf{#1})} % Affiche le titre entre parenthèses et en gras s'il est fourni
  . % Point à la fin
}
{
  \end{mdframed}
}


\theoremstyle{plain}
\newtheorem*{remark}{Remarque}
\newtheorem*{proposition}{Proposition}
\newtheorem*{lemma}{Lemme}
\newtheorem*{prop}{Propriété}
\newtheorem*{corollary}{Corollaire}
\newtheorem*{proof}{Démonstration}
\newtheorem*{example}{Exemple}


% OTHERS 


\newlength\tindent
\setlength{\tindent}{\parindent}
\setlength{\parindent}{0pt}
\renewcommand{\indent}{\hspace*{\tindent}}


\usepackage{fancyvrb}
\usepackage{tikz-cd} 
\usepackage{amsmath}
\usepackage{mathrsfs}  
\usepackage{amssymb}
\usepackage{tkz-graph}
\usepackage{caption}
\usepackage{multicol}
\usepackage{listings} % Importation du package listings
\usepackage{xcolor} % Pour ajouter de la couleur

\setlength{\columnsep}{1cm} % Espace entre les colonnes


% Définition du style de l'en-tête
\pagestyle{fancy}
\fancyhf{} % Nettoyer les en-têtes et pieds de page

% Gauche : Nom de la section
\fancyhead[L]{\nouppercase{\leftmark}}

% Droite : Logo
\fancyhead[R]{\includegraphics[width=2cm]{./images/logo_jfc.png}}

% Ligne sous l'en-tête
\renewcommand{\headrulewidth}{0.4pt}

\setlength{\headheight}{1.5cm} % Ajuste la hauteur de l'en-tête




\setlength{\headheight}{52.87724pt}  % Augmenter la hauteur de l'en-tête
\addtolength{\topmargin}{-10.19812pt}  % Réduire l'espace en haut de la page


\begin{document}

% ==================================================================================================================================
% TITLEPAGE 

\begin{titlepage}
    \begin{center}
        \includegraphics[width=0.4\textwidth]{./images/logo_arduino.png} \\[1cm]

        {\Huge \textbf{Rapport Mini Projet Arduino}} \\[1cm]

        {\Large \textbf{ArdyWeather}} \\[2cm]

        {\textbf{UE : Architecture des Ordinateurs}} \\[2cm]

        \textbf{Auteur :} Axel PIGEON \\[0.5cm]
        \textbf{Université :} INU Champollion \\[0.5cm]
        \textbf{Date :} \today \\[2cm]

        \vfill

        {\large Rapport dans le cadre d'un projet de développement en Arduino ayant pour but 
        l'initiation au développement embarqué et la communication entre différentes architectures 
        (Arduino, RaspberryPI, etc...)}
    \end{center}
\end{titlepage}


\tableofcontents

% ==================================================================================================================================
% Introduction 

\justify

\subsection{Introduction}


Ici est présenté le projet d'Architecture des Ordinateurs 2. Le projet consiste à développer un outil ou un jeu 
utilisant des connexion arduino/arduino ou arduino/raspberry. 

Pour celui-ci, nous allons essayer de faire une station météo composée de deux modules : 
\begin{itemize}
    \item \textbf{Le capteur : } composé d'une carte arduino munie de capteurs (ex : température, humidité...) et d'un émetteur/récepteur 
    radio pour envoyer les relevés à la station fixe. 
    Pour le capteur, on utilise un {capteur de température et d'humidité Grove} permettant 
    une connexion facile à la carte. 
    \item \textbf{La station fixe : } Composée d'un arduino recevant les données émises par le capteur et les 
    sauvegarde. 
\end{itemize}

Toutes les informations techniques relatives au projet (code, bibliothèques) sont disponibles sur le github : 
\begin{center}
    \href{https://github.com/winston2968/ArdyWeather.git}{https://github.com/winston2968/ArdyWeather.git}
\end{center}

L'objectif est de programmer la station pour qu'elle puisse interragir avec plusieurs 
stations. 


\newpage 

% ==================================================================================================================================
% Présentation et principe général

\section{Présentation et Principe général}

Le projet dispose de deux modules, le \texttt{main\_station} qui gère le protocole radio côté station et 
la gestion des données reçues par les capteurs et le \texttt{main\_sensor} qui gère le protocole radio côté 
capteur ainsi que la récolte et l'encodage des données. Attardons nous sur le protocole radio. 

\subsection{Structure des paquets}

La bibliothèque \texttt{VirtualWire} permet d'envoyer et de recevoir des données via un signal radio.
Malgré sa rusticité, elle est capable de différencier le signal radio émis par l'Arduino de celui émis par une 
autre Arduino. Autrement dit, on ne peut pas recevoir les messages que l'on envoie. 
Cette bibliothèque ne permet d'envoyer "que" 27 octets par émissions. Nous devons donc dimensionner les datagrammes en fonction. 
Dans un datagramme (ou paquet) nous devons donc envoyer : 
\begin{itemize}
    \item \textbf{Une séquence propre au protocole} pour différencier nos paquets d'autres émissions ambiantes. 
        (\emph{Nous choisirons pour cela deux caractères en début de datagramme : 'A' et 'W'. })
    \item \textbf{Un identifiant unique pour l'émeteur} du paquet. 
        (\emph{Un nombre entier correspondant au numéro du module. La station (qui reçoit les données 
        des capteurs) aura le nnuméro '0'.})
    \item \textbf{Un identifiant unique pour le destinataire} du paquet. 
        (\emph{Idem que pour les émetteurs.})
    \item \textbf{Le type du paquet} soit des données ou simplement un paquet d'acquittement. 
        (\emph{On prendra 'D' pour un paquet de données et 'A' pour une trame d'acquittement. })
    \item \textbf{Le type des données envoyée} le cas échéant, pour différencier les données de température et d'humidité. 
        (\emph{De même que précédement, on choisira 'T' pour température ou 'H' pour humidité. })
    \item \textbf{Le nombre de paquets déjà envoyés} : pour permettre de détecter un problème de réception. 
        (\emph{Un entier encodé avec son caractèse ASCII.})
    \item \textbf{Le nombre de paquets reçus} : pour informer l'émetteur de la bonne réception des paquets. 
        (\emph{Un entier encodé avec son caractèse ASCII.})
    \item \textbf{Deux octets de checksum }: pour être capable de détecter des paquets détériorés. 
        (\emph{Calculés à l'aide d'une simple somme modulaire.})
    \item \textbf{Les données} : température, humidité ou rien si c'est un paquet d'acquittement. 
        (\emph{Encodées d'une certaine façon que nous détaillerons plus tard}. )
\end{itemize}

Ainsi, un datagramme classique aura le format suivant : 
    \begin{center}
        \texttt{\{'A', 'W', Emetteur, Destinataire, Type de Paquet, Type de Données, Nb Envoyés, Nb ACK, CHKSM1, CHKSM2, Datas\}}
    \end{center}
Par exemple le paquet \texttt{'A','W','1','0','D','H','1','0',... }, est un datagramme envoyé 
par le module 1 à la station (module 0). C'est un paquet de données d'humidité. Le paquet porte le numéro 1 (donc la station
devra acquitter avec une trame d'acquittement et un numéro ACK de 1). On peut voir que la station n'a envoyé aucun paquet au 
module numéro 1. 
On peut voir, qu'une fois l'en-tête du paquet remplie, il ne nous reste "que" 16 octets pour les données dans un datagramme. 

\subsection{Encodage des données}

Les données de température et d'humidité relevés par les capteurs des modules sont des numbres flottant. 
Dans le module relevant les données, on les stocke donc sous forme de tableau de flottant, \texttt{float temp\_table[]}
et \texttt{float hum\_table[]} dans le code. 
Pour des données d'humidité et de température, nous n'avons besoin d'une précision à seulement deux décimales. 
Or les flottant sont encodés sur 8 octets en Arduino. Si on envoyait les flottant directement dans les datagrammes, 
il nous faudrait envoyer un nombre conséquent de paquets dont le champ de données serait à moitié vide. 

Ainsi, nous avons choisis d'encoder ces nombres flottant sur seulement 3 octets. Pour cela, nous les multiplierons 
par 100 pour obtenir des entiers. Entier, que l'on encodera ensuite sur 3 octets, chaque octet définis par des opérations 
arithmétiques (voir fonction \texttt{convert\_float\_table\_to\_ASCII} du fichier \texttt{sensors.ino} du module \texttt{main\_sensor}). 
Ces opérations ne nous permettent que d'envoyer les flottant entre $0$ et $85.54$. Donc on ne peut pas relever des températures 
négatives...

\subsection{Principe et fonctionnement}

L'émission des datagrammes de données sont déclenchés par les \textbf{capteurs} et réceptionnés par la \textbf{station}. 

\vspace{0.3cm}

\textbf{Côté capteur}, celui-ci relève en permanence la température et l'humidité de son environnement et les stocke dans deux tables 
\texttt{temp\_table} et \texttt{hum\_table} comme énoncé précédement. 
Une fois ces tables suffisament remplies (on a choisis un nombre arbitraire de 64 relevés correspondant à $13 \times 2$ émissions radio), 
le capteurs déclenche la séquence d'envoi de données à la station. 
Pour cela, il dispose de la fonction \texttt{update\_temp\_hum} déclenchée par un timer qui relève les données 
en permanence et la fonction \texttt{build\_merged\_table} qui permet de construire la \texttt{char merged\_table[]}. 
Cette table de données permet de faire un tampon entre l'émission radio et le stockage des données en cours de relevé. 

En effet, la séquence d'envoi peut être longue en fonction du temps que met la station à recevoir et acquitter les datagrammes. 
Il est donc indispensable de libérer les tables de relevés au cours de cet intervalle pour permettre de continuer 
à enregistrer la température et l'humidité. 

Ainsi, le capteur dispose de suffisament de temps (exactement 26 fois l'intervalle de relevés) pour envoyer les données. 
Passé ce temps là, la \texttt{merged\_table} est écrasée par les nouvelles données à envoyer. 

L'envoi des datagrammes est déclenché par un timer. A chaque activation, le capteur vérifie dans la fonction 
\texttt{send\_datagram} si il est en cours de séquence d'envoi et si tous les datagrammes précédents ont étés acquittés. 
Si tel est le cas, il récupère les données à envoyer en fonction de la valeur de \texttt{nb\_tables\_stack} pointant sur 
l'indice de départ de la séquence de \texttt{merged\_table} à envoyer. 
Dans le cas contraire (si le précédent datagramme n'a pas été acquitté), il le renvoie via la variable \texttt{char last\_datagram[]}
actualisée à chaque envoi de nouveau datagramme. 
A noter qu'il n'est pas possible d'avoir envoyé plusieurs trames non acquittées, puisque le capteur attend toujours l'acquittement 
de la précédente pour envoyer la suivante. 

\vspace{0.3cm}

\textbf{Côté station}, celle-ci attend en permanence la réception de datagrammes via son module radio. 
Une fois un datagramme réceptionné correspondant au protocole développé, elle calcule vérifie que les octets 
des champs checksum correspondent bien au calcul de la trame reçue. En cas d'égalité, elle acquitte le datagramme à l'émetteur et 
sauvegarde les données dans \texttt{temp\_storage} et \texttt{hum\_storage} en fonction de leur type. 




% ==================================================================================================================================
% Développement

\section{Développement}

Dans cette partie, nous allons détailler brièvement le processus de développement du projet ainsi que les outils 
utilisés. 

\subsection{Structure et choix}

Tout d'abord, l'utilisation d'Arduino de type MEGA disposant d'un mémoire RAM de 8ko a grandement facilité 
la gestion de la mémoire pour le stockage d'importantes quantités de données de température et d'humidité. 
Dans un second temps, le choix de ces modules radio assez peu perfectionnés nous a poussé à repenser complètement 
un protocole de communication fiable et solide. L'acquittement de chaque datagramme de données n'étant pas la solution 
la plus optimale, elle reste quand même la plus fiable et simple à implémenter. 
L'ajout de checksum rajoute une couche de sécurité supplémentaire permettant d'isoler tout paquet endommagé. 
Enfin, l'ajout d'un dispositif de chiffrement des datagrammes aurait pu être une bonne idée mais les données envoyées 
n'étant pas sensibles, ce n'était pas une priorité du projet. 

\vspace{0.3cm}

Côté techniques et outils de développement, le logiciel \href{https://www.arduino.cc/en/software/}{ArduinoIDE} 
nous a permis de développer le projet de A à Z. Sa fiabilité et sa facilité d'utilisation en a fait un outil de choix. 
De plus, sa capacité à inclure plusieurs fichiers dans un même sketch (projet à téléverser dans l'Arduino) a été 
pratique pour fragmenter le code et ne pas avoir à traiter un fichier de plus de 1000 lignes. 
Pour ce qui est de la gestion des différentes versions du projet, nous avons utilisé GitHub pour le versionnage et la 
sauvegarde. Le projet est d'ailleurs toujours disponible sur le site via le lien en introduction. 

\subsection{Améliorations}

Même si le projet final reste assez complet, il est nécessaire d'aborder les différentes fonctionnalités manquantes. 
Tout d'abord, un lecteur avisé du code remarquera rapidement que les capteurs de température et d'humidité ne sont 
utilisés. En effet, il nous a été impossible de les faire marquer avec la bilibothèque proposée par le fabriquant. 
Nous avons donc choisi de générer aléatoirement des valeurs pour simuler des relevés et pouvoir tester la gestion des données. 

\vspace{0.3cm}

D'autre part, le protocole de communication entre modules n'est pas des plus perfectionnés. Il serait plus optimal 
que les capteurs soient capables d'envoyer plusieurs datagrammes de suite et de recevoir une seule trame 
d'acquittement pour tous les paquets envoyés. Cela réduirait beaucoup le flux de données en transit et le 
nombre d'acquittements à envoyer pour la station principale. 

\vspace{0.3cm}

Enfin, il aurait été préférable d'ajouter une option de visualisation des données via un lien station/raspberry pi 
pour un affichage sur ordinateur. L'ajout d'affichers sur les modules ou la station aurait été préférable pour être 
capable de visualiser ce que font les micro-contrôleurs. 


\subsection{Processus de Test}

Dans le \href{https://github.com/winston2968/ArdyWeather.git}{GitHub}, plusieurs fichiers permettent de tester 
les différentes fonctionnalités du projet. 
Tout d'abord, les sketchs \texttt{main\_sensor} et \texttt{main\_station} contiennent tout le code des différents 
modules du projet. Il sont fonctionnels et leur lancement sur deux cartes ArduinoMEGA permettront d'avoir une bonne 
vision de ce qui est possible. 
De plus, l'ajout d'un sketch supplémentaire : \texttt{testing\_station} permet de tester uniquement le module station pour vérifier qu'il est 
bien capable de recevoir des datagrammes de différents autres modules. 



% ==================================================================================================================================
% Conclusion

\section{Conclusion}



\end{document}
