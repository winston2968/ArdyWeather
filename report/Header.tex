% Use documentclass{article}


\usepackage[english]{babel}
\usepackage[T1]{fontenc}
\usepackage{ragged2e}
\usepackage{amsfonts}
\usepackage{systeme}
\usepackage{amsmath}
\usepackage{amssymb}
\usepackage{comment}
\usepackage{multicol}
\usepackage{lipsum} 
\usepackage{graphicx}
\usepackage{stmaryrd}
\usepackage{systeme}
\usepackage{wrapfig}
\usepackage{colortbl}
\usepackage{cellspace}
\usepackage{stmaryrd}
\usepackage{ntheorem}
\usepackage{lmodern}
\usepackage{mathtools}
\usepackage{wrapfig}
\usepackage{ragged2e}
\usepackage{tabularx}
\usepackage{titlepic}
\usepackage{fancyhdr}
\usepackage{caption}
\usepackage{xcolor} % pour les couleurs
\usepackage[linkbordercolor=white]{hyperref} % après avoir chargé xcolor
\usepackage{systeme}
\usepackage[T1]{fontenc}
\usepackage{lmodern}
\usepackage{listings}
\usepackage{tikz}
\usepackage{mdframed}
\usepackage{xparse} % Nécessaire pour définir des environnements avec arguments optionnels
\usepackage[top=3cm, bottom=3cm, left=3.5cm, right=3.5cm]{geometry}

% \setcounter{secnumdepth}{-1} % Désactive le compteur des parties

% PAGE SETTINGS

% \justifying

% \setlength{\columnseprule}{1pt}
% \def\columnseprulecolor{\color{black}}

\newcolumntype{C}{>{$\displaystyle}Sc<$}
\cellspacetoplimit=5pt
\cellspacebottomlimit=5pt


% MATHS SHORTHANDS

\newcommand{\C}{\mathbb{C}}
\newcommand{\R}{\mathbb{R}}
\newcommand{\Q}{\mathbb{Q}}
\newcommand{\Z}{\mathbb{Z}}
\newcommand{\N}{\mathbb{N}}
\newcommand{\U}{\mathbb{U}}
\newcommand{\K}{\mathbb{K}}
\newcommand{\M}{\mathcal{M}}
\newcommand{\B}{\mathcal{B}}

\renewcommand{\epsilon}{\varepsilon}
\renewcommand{\phi}{\varphi}
\renewcommand{\rho}{\varrho}

% MOD NOTATION

\theorembodyfont{\upshape}

% Définir un environnement pour encadrer les définitions avec un titre
\NewDocumentEnvironment{definition}{O{}}
{
  \begin{mdframed}[linewidth=0pt,linecolor=gray,backgroundcolor=gray!10,roundcorner=5pt]
  \textbf{Définition}%
  \IfNoValueTF{#1}{}{~(\textbf{#1})} % Affiche le titre entre parenthèses et en gras s'il est fourni
  . % Point à la fin
}
{
  \end{mdframed}
}

% Définir un environnement pour encadrer les théorèmes avec un titre
\NewDocumentEnvironment{theorem}{O{}}
{
  \begin{mdframed}[linewidth=1pt,linecolor=darkgray,backgroundcolor=darkgray!10,roundcorner=5pt]
  \textbf{Théorème}%
  \IfNoValueTF{#1}{}{~(\textbf{#1})} % Affiche le titre entre parenthèses et en gras s'il est fourni
  . % Point à la fin
}
{
  \end{mdframed}
}

% Définir un environnement pour encadrer les corollaires
\NewDocumentEnvironment{corollary}{O{}}
{
  \begin{mdframed}[linewidth=1pt,linecolor=gray,backgroundcolor=gray!10,roundcorner=5pt]
  \textbf{Corollaire}%
  \IfNoValueTF{#1}{}{~(\textbf{#1})} % Affiche le titre entre parenthèses et en gras s'il est fourni
  . % Point à la fin
}
{
  \end{mdframed}
}
				

\theoremstyle{plain}
\newtheorem*{remark}{Remarque}
\newtheorem*{proposition}{Proposition}
\newtheorem*{lemma}{Lemme}
\newtheorem*{prop}{Propriété}
\newtheorem*{proof}{Démonstration}
\newtheorem*{example}{Exemple}


% OTHERS 


%\newlength\tindent
%\setlength{\tindent}{\parindent}
%\setlength{\parindent}{0pt}
%\renewcommand{\indent}{\hspace*{\tindent}}

% Put begin and end document 
