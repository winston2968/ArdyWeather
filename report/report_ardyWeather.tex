\documentclass{article}


\input{Header.tex}
\usepackage[utf8]{inputenc}


\begin{document}


% ==================================================================================================================================


\begin{center}
    \textbf{ - Projet Architecture des Ordinateurs - }

    \Large{\textbf{ArdyWeather}}
\end{center}

\rule{\linewidth}{1.5pt}


% ==================================================================================================================================

\justify

\subsection{Introduction}

Ici est présenté le projet d'Architecture des Ordinateurs 2. Le projet consiste à développer un outil ou un jeu 
utilisant des connexion arduino/arduino ou arduino/raspberry. 

Pour celui-ci, nous allons essayer de faire une station météo composée de deux modules : 
\begin{itemize}
    \item \textbf{Le capteur : } composé d'une carte arduino munie de capteurs (ex : température, humidité...) et d'un émetteur/récepteur 
    radio pour envoyer les relevés à la station fixe. 
    \item \textbf{La station fixe : } Composée d'un arduino recevant les données émises par le capteur et les envoie en filaire 
    à un carte type raspberry pour un affichage sur écran/interface web. 
\end{itemize}

Toutes les informations techniques relatives au projet (code, bibliothèques) sont disponibles sur le github : 
\begin{center}
    \href{https://github.com/winston2968/ArdyWeather.git}{https://github.com/winston2968/ArdyWeather.git}
\end{center}


\end{document}
